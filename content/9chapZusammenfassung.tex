\chapter{Zusammenfassung}
\label{chap:zusammenfassung}
In dieser Arbeit wurde ein Dashboard implementiert, das speziell für den Einsatz in einer \gls{dfir} Umgebung für \gls{open-ran} konzipiert wurde. Ziel war es, ein Werkzeug zu schaffen, das sowohl die Durchführung von Pentesting-Szenarien unterstützt als auch die Bewertung von Schwachstellen in \gls{open-ran}-Systemen ermöglicht. Die Weiterentwicklung basierte auf der Vorarbeit aus einer vorherigen Forschung und der Integration der empirischen Methode \gls{acema}, die durch ihre Fähigkeit zur Analyse und Visualisierung von Schwachstellen wesentlich zur Funktionalität des Dashboards beiträgt. Die Ergebnisse belegen die technische und wissenschaftliche Relevanz des Dashboards für den Einsatz in einer \gls{dfir} Umgebung.

Die Implementierung des Dashboards wurde erfolgreich abgeschlossen, wobei alle Meilensteine planmäßig erreicht wurden. Zu den zentralen Ergebnissen zählt die Integration von \gls{acema}, die durch die Nutzung von \gls{cvss}-Daten eine präzise Bewertung von Schwachstellen in \gls{open-ran} Umgebungen ermöglicht. Die dynamischen Visualisierungen unterstützen die Nutzer dabei, Beziehungen zwischen spezifischen Angriffe und Techniken aus etablierten Frameworks wie \gls{mitre} \gls{attack} schnell zu erkennen.

Die Analyse der von \gls{acema} generierten Diagramme zeigte, dass \gls{open-ran}-Systeme durch standardmäßige Netzwerkanbindung ein erhöhtes Angriffspotenzial aufweisen. Beispielsweise zeigte die Analyse, dass hochkritische Angriffsmuster wie \textit{Remote Code Execution} in \gls{open-ran} Umgebungen besonders relevant sind, da sie über das Netzwerk ausgenutzt werden können und schwerwiegende Auswirkungen auf die Integrität, Verfügbarkeit und Vertraulichkeit der Systeme haben. Die Ergebnisse verdeutlichen auch, dass die meisten relevanten Schwachstellen mit einem mittlerem oder hohem Schweregrad bewertet werden und spezifische Bedrohungen wie die Ausnutzung von Container-Service-Konten oder privilegierten \glspl{vm} besonders gefährlich sind. Dies unterstreicht die Bedeutung einer regelmäßigen Bedrohungsanalyse und einer Implementierung nach dem \textit{security-by-design}-Prinzip in Kubernetes-basierten Umgebungen.

Mit der Entwicklung dieses Dashboards wurde ein wichtiger Beitrag zur Sicherheitsforschung im Bereich \gls{open-ran} geleistet, indem ein praktisches \textit{Open-Source}\footnote{Siehe Anhang \ref{app:sourcecode}.}-Werkzeug für die Visualisierung und Analyse von Schwachstellen als bereitstellt wird. Die Relevanz einer empirischen Analyse und Kategorisierung von Schwachstellen in unterschiedlichen technologischen Kontexten wird durch die Studien von \autocite{mazuera-rozoAndroidOSStack2019} und \autocite{klementSecuring6GTransition2024} verdeutlicht. Die Ergebnisse dieser Arbeit ergänzen die Studie von \citeauthor{klementSecuring6GTransition2024} und zeigt, wie wissenschaftliche Methoden in praxisnahe Anwendungen überführt werden können.

Zusammenfassend lässt sich sagen, dass das Dashboard ein effektives Werkzeug für die Sicherheitsanalyse in \gls{open-ran} Umgebungen sein kann. Die gewonnenen Erkenntnisse bieten nicht nur praktische Ansätze zur Verbesserung der Sicherheit von \gls{open-ran}, sondern dienen auch als Grundlage für zukünftige Forschung und Entwicklung in \gls{dfir} Umgebungen.