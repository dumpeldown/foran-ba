\chapter*{Kurzfassung}
\label{chap:kurzfassung}
%
Eine Kurzfassung (wenn verlangt) in Deutsch und/oder in Englisch (\emph{Abstract}) umfasst auf etwa~1/2 bis~1 Seite die Darstellung der Problemstellung, der angewandten Methode(n) und des wichtigsten Ergebnisses.
\par
Wie man ein gelungenes Abstract verfasst, erfahren Sie in den Seminaren oder der Beratung des Schreibzentrums der Kompetenzwerkstatt\footnote{\href{https://www.th-koeln.de/schreibzentrum}{https://www.th-koeln.de/schreibzentrum}}.
\par
Schlagwörter/Schlüsselwörter: evtl. Angabe von~3 bis~10 Schlagwörtern.

Es wird empirisch aufgezeigt, welche Angriffstechniken besonders risikobehaftet sind, beziehungsweise aus der Perspektive eines Angreifers ein höheres Potenzial bieten. Dasselbe Ergebnis wurde auch in Bezug auf O-RAN erreicht. Über empirische und statistische Methoden werden O-RAN Komponenten identifiziert, die bei Ausnutzung von spezifischen Techniken gefährdet sind.