\chapter*{Kurzfassung}
\label{chap:kurzfassung}
\glsreset{dfir}
\glsreset{open-ran}
Ein in dieser Arbeit entwickeltes Dashboard wurde speziell für den Einsatz in \gls{dfir}-Umgebungen für \gls{open-ran} entworfen, um Pentesting-Szenarien zu unterstützen und Schwachstellen präzise zu bewerten. Die Integration einer empirischen Methode und die Nutzung von CVSS-Daten ermöglichen detaillierte Analysen und Visualisierungen. Das Dashboard bietet ein Werkzeug zur Visualisierung und Analyse von Schwachstellen im stark virtualisierten Open RAN Umfeld. Es wird empirisch aufgezeigt, welche Angriffstechniken besonders risikobehaftet sind, beziehungsweise aus der Perspektive eines Angreifers ein höheres Potenzial bieten. Dasselbe Ergebnis wurde auch in Bezug auf O-RAN erreicht. Über empirische und statistische Methoden werden O-RAN Komponenten identifiziert, die bei Ausnutzung von spezifischen Techniken gefährdet sind. Das Ergebnis dieser Arbeit stellt einen wichtigen Beitrag zur Sicherheitsforschung dar und bietet zugleich Ansätze für zukünftige Entwicklungen und Untersuchungen in \gls{dfir}-Kontexten.
\par Schlüsselwörter: Open RAN, O-RAN, DFIR, 5G, Dashboard

