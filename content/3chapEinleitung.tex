\chapter{Einleitung}
\label{chap:Einleitung}

\par
Die Einführung des 5G-Netzwerks stellt einen bedeutenden Schritt in der Telekommunikationslandschaft dar und eröffnet durch die erhöhte Flexibilität und Geschwindigkeit neue Anwendungsfelder. Eine zentrale Entwicklung hierbei ist das Konzept von \gls{open-ran}, welches durch die offene Spezifikation und modulare Architektur das herkömmliche RAN-Design revolutioniert. Open RAN ermöglicht eine Interoperabilität zwischen verschiedenen Herstellern und eine flexiblere Verwaltung der Netzwerkkomponenten, was den Netzwerkbetreibern potenzielle Kosteneinsparungen und eine größere Anpassungsfähigkeit bietet. Die Abhängigkeit von proprietären Herstellen wie Huawei, Ericsson and Nokia fallen dabei weg \autocite{NokiaEricssonUnd}. Die \gls{oran-alliance}, ein Konsortium aus Telekommunikationsanbietern, Herstellern und Forschungsinstitutionen, treibt diesen Ansatz voran und entwickelt Standards und Architekturen, die speziell für 5G und zukünftig auch für 6G ausgelegt sind. Diese Standards sind auf die Entkopplung von Hardware und Software ausgelegt und fördern die Virtualisierung von RAN-Komponenten, um so die Netzwerksicherheit zu verbessern und zugleich die Flexibilität der Netzbetreiber zu erhöhen \autocite{5GFORAN}.
\par
Das Forschungsprojekt \textit{\gls{foran}}, eine Kooperation zwischen der Technischen Hochschule Köln und PROCYDE GmbH, widmet sich den sicherheitstechnischen Herausforderungen im Open RAN Umfeld, insbesondere im Kontext der \gls{dfir}. Ziel des Projekts ist die Entwicklung von Methoden zur Analyse, Behandlung und Behebung von Sicherheitsvorfällen in Open RAN-Netzwerken. Hierbei wird der gesamte Lebenszyklus eines Sicherheitsvorfalls berücksichtigt: von der Erkennung über die Analyse bis hin zur forensischen Nachverfolgung. 
\par Das Projekt liefert wissenschaftliche Erkenntnisse zur Sicherheit von \gls{open-ran}-Systemen, indem neue Ansätze zur Analyse und Behandlung von Sicherheitsvorfällen entwickelt oder auf existierender Forschung aufgebaut wird. Die Verknüpfung empirischer Daten mit bewährten Sicherheitsmodellen wie dem \gls{mitre} \gls{attack}-Framework schafft eine wissenschaftliche Basis zur Bewertung von Risiken. Die Forschung trägt dazu bei, das Verständnis für die Bedrohungslandschaft in virtualisierten Mobilfunknetzen zu vertiefen und liefert Ansätze für die Weiterentwicklung von sicheren Implementierungen der \gls{oran}-Spezifikationen. Der wissenschaftliche Wert des Projekts liegt insbesondere in der Übertragbarkeit der entwickelten Methoden auf andere virtualisierte oder containerisierte Systeme sowie der Schaffung einer Grundlage für weiterführende Arbeiten im Bereich 5G- und 6G-Sicherheit.
\par Das Forschungsprojekt \gls{foran} ist in zwei Teilprojekte unterteilt: \gls{foran}-ATTACK und \gls{foran}-\gls{dfir}. Im Rahmen von \gls{foran}-ATTACK wird ein System entwickelt, das gezielte Angriffssimulationen auf \gls{open-ran}-Komponenten ermöglicht, um spezifische Angriffsspuren zu generieren. Diese Angriffsspuren können dann mit - im Teilprojekt \gls{foran}-\gls{dfir} - entwickelten Methoden identifiziert und behandelt werden. Dabei liegt der Fokus auf der Anpassung etablierter \gls{dfir}-Frameworks für die spezifischen Anforderungen im Open RAN-Bereich \autocite{5GFORAN}. Eine sehr wichtige Rolle bei der Kategorisierung von Angriffen und der Bewertung von Schwachstellen spielt die Arbeit des \gls{mitre}. Die \gls{mitre}-Corporation stellt unter anderem mit dem \gls{attack}-Framework eine umfassende strukturierte Sammlung von Angriffstechniken und -taktiken bereit \autocite{MITREATTCK} \autocite{SolvingProblemsSafer2024}.
\par
In dieser Arbeit liegt der Fokus auf der Entwicklung und Implementierung eines Dashboards, welches als zentrales Tool für die Visualisierung und Analyse von Angriffssimulationen fungiert. Das Dashboard stellt eine zentrale Komponente für das ATTACK-Teilvorhaben von \gls{foran} dar. Zu den spezifischen Anforderungen zählen die Darstellung von Angriffspfaden und die Integration des \gls{cvss} zur Bedrohungsbewertung. Außerdem wurde speziell für die Kategorisierung der Angriffe eine abgewandelte Version der \gls{mitre} \gls{attack}-Matrix erstellt und die Daten empirisch eingeordnet. Dies unterstützt eine analytische Herangehensweise an die Bewertung der Angriffe \gls{open-ran}\autocite{dieterichDevelopmentAdversarySimulation}.
\par
Diese Arbeit implementiert die Methoden, die \citeauthor{klementSecuring6GTransition2024} in ihrem Paper \textit{\gls{acema-full}} beschreiben, um Schwachstellen in Open RAN-Komponenten anhand empirischer Daten zu bewerten. Es wird hierbei keine Bewertung aller möglichen Angriffe vorgenommen, sondern die durch das Teilvorhaben \gls{foran}-ATTACK gefundenen und durch das ATTACK-Tool simulierten Angriffe auf eine Referenzimplementierung von \gls{open-ran} der \gls{oran} Software Community in Version \textit{I} bewertet \todo{Was ist die korrekt Release Version?}. Die Bewertung nach dem \gls{cvss} wird in der Übersicht der einzelner Angriffe und in der modifizierten \gls{mitre}-Matrix dargestellt \autocite{dieterichDevelopmentAdversarySimulation,klementSecuring6GTransition2024}.
