\chapter{Einleitung}
\label{chap:Einleitung}

\par
Die Einführung des 5G-Netzwerks stellt einen bedeutenden Schritt in der Telekommunikationslandschaft dar und eröffnet durch die erhöhte Flexibilität und Geschwindigkeit neue Anwendungsfelder. Eine zentrale Entwicklung hierbei ist das Konzept von \gls{open-ran}, welches durch die offene Spezifikation und modulare Architektur das herkömmliche RAN-Design revolutioniert. Open RAN ermöglicht eine Interoperabilität zwischen verschiedenen Herstellern und eine flexiblere Verwaltung der Netzwerkkomponenten, was den Netzwerkbetreibern potenzielle Kosteneinsparungen und eine größere Anpassungsfähigkeit bietet. Die Abhängigkeit von proprietären Herstellen wie Huawei, Ericsson and Nokia fallen dabei weg \cite{NokiaEricssonUnd}. Die \gls{oran}, ein Konsortium aus Telekommunikationsanbietern, Herstellern und Forschungsinstitutionen, treibt diesen Ansatz voran und entwickelt Standards und Architekturen, die speziell für 5G und zukünftig auch für 6G ausgelegt sind. Diese Standards sind auf die Entkopplung von Hardware und Software ausgelegt und fördern die Virtualisierung von RAN-Komponenten, um so die Netzwerksicherheit zu verbessern und zugleich die Flexibilität der Netzbetreiber zu erhöhen \cite{5GFORAN}.

\par
Das Forschungsprojekt \textit{5G-\gls{foran}}, eine Kooperation zwischen der Technischen Hochschule Köln und PROCYDE GmbH, widmet sich den sicherheitstechnischen Herausforderungen im Open RAN Umfeld, insbesondere im Kontext der \gls{dfir}. Ziel des Projekts ist die Entwicklung von Methoden zur Analyse, Behandlung und Behebung von Sicherheitsvorfällen in Open RAN-Netzwerken. Hierbei wird der gesamte Lebenszyklus eines Sicherheitsvorfalls berücksichtigt: von der Erkennung über die Analyse bis hin zur forensischen Nachverfolgung.
Das Forschungsprojekt 5G-\gls{foran} ist in zwei Teilprojekte unterteilt: 5G-\gls{foran}-ATTACK und 5G-\gls{foran}-\gls{dfir}. Im Rahmen von 5G-\gls{foran}-ATTACK wird ein System entwickelt, das gezielte Angriffssimulationen auf \gls{open-ran}-Komponenten ermöglicht, um spezifische Angriffsspuren zu generieren. Diese Angriffsspuren können dann mit - im Teilprojekt 5G-FORAN-DFIR - entwickelten Methoden identifiziert und behandelt werden. Dabei liegt der Fokus auf der Anpassung etablierter \gls{dfir}-Frameworks für die spezifischen Anforderungen im Open RAN-Bereich, um Sicherheitsvorfälle effizient erkennen und bearbeiten zu können \cite{5GFORAN}.

\par
In dieser Arbeit liegt der Fokus auf der Entwicklung und Implementierung eines Dashboards, welches als zentrales Tool für die Visualisierung und Analyse von Angriffsdaten fungiert. Das Dashboard stellt eine zentrale Komponente für das ATTACK-Framework von 5G-FORAN dar und ermöglicht es, Angriffssimulationen zu visualisieren. Zu den spezifischen Anforderungen zählen die Darstellung von Angriffspfaden, die Integration des \gls{cvss} zur Bedrohungsbewertung sowie die Anpassung des MITRE-\gls{attack}-Frameworks an den spezifischen Anwendungsfall und die Darstellung der angepassten MITRE-Matrix zur Identifizierung und Analyse potenzieller Angriffstechniken  \gls{open-ran}\cite{dieterichDevelopmentAdversarySimulation}.