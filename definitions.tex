%%%%%%%%%%%%%%%%%%%%%%%%%%%%%%%%%%%%%%%%%%%%%%%%%%%%%%%%%%%%%%%%%%%%%%%%%%%%%%%
% Globale Definitionen
%%%%%%%%%%%%%%%%%%%%%%%%%%%%%%%%%%%%%%%%%%%%%%%%%%%%%%%%%%%%%%%%%%%%%%%%%%%%%%%


%=== pdf Metadaten ============================================================
\hypersetup{
	pdfauthor={Jurek Miklas Jesse},
	pdftitle={Implementierung eines Dashboard für Pentesting in einer Digital Forensik and Incident Response (DFIR) Umgebung für Open RAN},
	pdfsubject={Bachelorarbeit},
	pdfkeywords={
		Open RAN,
		O-RAN,
		DFIR,
		5G,
		Dashboard
	},
	bookmarksnumbered=true,
	pdfstartview=FitH,
	hidelinks,
}

%=== Kopf-/Fusszeile definieren ===============================================
\clearpairofpagestyles
\ohead[]{\headmark}
\ofoot[\pagemark]{\pagemark}

%=== Farben definieren ========================================================
\definecolor{THRed}{RGB}{207,24,32}
\definecolor{THOrange}{RGB}{236,101,37}
\definecolor{THPurple}{RGB}{175,54,140}

%=== Einstellungen für cref ===================================================
\newcommand{\crefpairconjunction}{ und~}
\newcommand{\crefrangeconjunction}{ bis~}
\crefname{figure}{Abbildung}{Abbildungen}
% Define custom colors
\definecolor{mygray}{rgb}{0.95,0.95,0.95}

% Define a custom environment for code blocks
\lstnewenvironment{code}[1][] % Optional argument for customization
{
  \lstset{
    backgroundcolor=\color{mygray},
    basicstyle=\ttfamily\scriptsize,
    breaklines=true,
    breakatwhitespace=true,
    frame=single,
    rulecolor=\color{gray},
    xleftmargin=5mm,
    numbers=left,
    numberstyle=\tiny,
    captionpos=b, % Position of the caption (b for bottom, t for top)
	float, % Hinzufügen der float-Option
    floatplacement=htb!,
    #1 % Allow overriding settings using optional arguments
  }
}
{}

%=== Einstellungen für plots ==================================================
\pgfplotsset{
	compat=newest,
	/pgf/number format/.cd,
	dec sep={\text{,}},
	1000 sep={\,},
}

%=== Einstellungen für listings ===============================================
\lstdefinestyle{myLaTeX}{
	basicstyle=\footnotesize\ttfamily,
	language=TeX,
	keywordstyle=\color{blue},
	frame=single,
	backgroundcolor=\color{gray!10},
	tabsize=2,
	morekeywords={
		lstdefinestyle,
		footnotesize,
		ttfamily,
		color,
	},
}

\lstdefinestyle{myBasic}{
	basicstyle=\footnotesize\ttfamily,
	frame=single,
	escapechar={|_},
	backgroundcolor=\color{white},
	keywordstyle=\color{black},
}

\lstset{style=myLaTeX}

\DeclareDelimFormat{multicitedelim}{\addcomma}
\DeclareDelimFormat{compcitedelim}{\addcomma}
\DefineBibliographyStrings{german}{%
  andothers = {{et al}\adddot}
}

\newcommand{\orana}{O-RAN Alliance}

%%%%%%% biblatex erlauben, im litverzeichnis die zeile überall zu brechen
% If you want to break on URL numbers
\setcounter{biburlnumpenalty}{9000}
% If you want to break on URL lower case letters
\setcounter{biburllcpenalty}{9000}
% If you want to break on URL UPPER CASE letters
\setcounter{biburlucpenalty}{9000}